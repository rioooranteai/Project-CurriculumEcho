\documentclass[12pt,a4paper]{article}
\usepackage[utf8]{inputenc}
\usepackage{geometry}
\usepackage{graphicx}
\usepackage{amsmath}
\usepackage{hyperref}
\usepackage{lipsum}

% Page layout
\geometry{top=1in, bottom=1in, left=1in, right=1in}

% Title settings
\title{\textbf{Deep Learning Course \\ Final Project Report}}
\author{Nama 1 (NIM)\\ Nama 2 (NIM)}
\date{\today}

\begin{document}

\maketitle
\tableofcontents
\newpage

% 1. Introduction
\section{Introduction}
In this section, introduce the project by providing an overview of the problem, the goals of the project, and its significance in the field of Deep Learning. Include:
\begin{itemize}
    \item Background and context of the project.
    \item Problem statement.
    \item Objectives of the research.
\end{itemize}

\noindent Example:
\lipsum[1] % Replace this with your content.

% 2. Related Works
\section{Related Works}
Discuss previous works and studies that relate to your project. Summarize how your project builds on or differs from these works.
\begin{itemize}
    \item Literature review of relevant papers and articles.
    \item Comparison of different approaches and their results.
    \item Justification for your chosen methodology.
\end{itemize}

\noindent Example:
\lipsum[2] % Replace this with your content.

% 3. Dataset and Material
\section{Dataset and Material}
Describe the dataset(s) used in the project, including details such as:
\begin{itemize}
    \item Source of the dataset.
    \item Preprocessing steps.
    \item Features and labels included in the data.
\end{itemize}

Additionally, list the tools, libraries, and frameworks used for model implementation and analysis.

\noindent Example:
\lipsum[3] % Replace this with your content.

% 4. Result and Discussion
\section{Result and Discussion}
Present the findings and results of your project. Include:
\begin{itemize}
    \item Performance metrics (e.g., accuracy, F1-score, confusion matrix).
    \item Visualization of results (e.g., charts, graphs, or tables).
    \item Discussion of the results, comparing them to related works.
\end{itemize}

\noindent Example:
\begin{itemize}
    \item \textbf{Accuracy:} 95.2\%
    \item \textbf{Precision:} 94.8\%
    \item \textbf{Recall:} 96.0\%
\end{itemize}

\begin{figure}[h!]
    \centering
    \includegraphics[width=0.8\textwidth]{example-result.png} % Replace with your image
    \caption{Example Result Visualization}
    \label{fig:result}
\end{figure}

% 5. Conclusion
\section{Conclusion}
Summarize the project findings, key takeaways, and the significance of the results. Include:
\begin{itemize}
    \item Recap of the objectives and how they were met.
    \item Key insights from the results.
    \item Future work and recommendations.
\end{itemize}

\noindent Example:
\lipsum[4] % Replace this with your content.

\newpage
\section*{References}
\bibliographystyle{plain}
% Add your bibliography entries here
% Example:
% \begin{thebibliography}{9}
% \bibitem{key1} Author Name, "Title of the Paper," Journal, Year.
% \end{thebibliography}

\end{document}